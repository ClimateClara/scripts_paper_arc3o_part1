This model was developed from scratch by Philipp Griewank during and after his PhD at Max Planck Institute of Meteorology from 2010-\/2014. Most elements of the model are described in my PhD thesis (A 1D model study of brine dynamics in sea ice, 2013, Hamburg) and in the publication \char`\"{}Insights into brine dynamics and sea ice desalination from a 1-\/D model study of gravity drainage\char`\"{} by Griewank \& Notz 2013 JGR: Oceans. Important changes from the version 1.0 to the previous version are:
\begin{DoxyItemize}
\item the switch to a harmonic mean permeability for gravity drainage
\item A few important bug fixes, which have a large affect on flushing
\end{DoxyItemize}

\hyperlink{SAMSIM_8f90}{SAMSIM.f90} is the root program of the SAMSIM, the 1D thermodynamic Semi-\/Adaptive Multi-\/phase Sea-\/Ice Model. The code is intended to be understandable and subroutines, modules, functions, parameters, and global variables all have doxygen compatible descriptions. However, in \hyperlink{SAMSIM_8f90}{SAMSIM.f90} only the testcase and description thread are specified, which are then passed on to \hyperlink{namespacemo__grotz}{mo\_\-grotz}, which is where most of the actual work is done, including timestepping.

WARNING: SAMSIM was developed and used for scientific purposes. It surely contains at least some undetected bugs, can easily be crashed by using non-\/logical input settings, and some of the descriptions and comments may be outdated. Always check the plausibility of the model results!

Getting started.
\begin{DoxyItemize}
\item A number of testcases are implemented in SAMSIM. Testcases 1, 2, 3, and 4 are intended as standard testcases which should give a first time user a feel for the model capabilities and serve as a basis to set up custom testcases. To familiarize yourself with the model I suggest running testcases 1-\/3 and plotting the output with the python plotting scripts provided. The details of each testcase are commented in \hyperlink{mo__init_8f90}{mo\_\-init.f90}, and each plot script begins with a list of steps required.
\end{DoxyItemize}

Running SAMSIM the first times.
\begin{DoxyItemize}
\item Make sure that all .f90 files are located in the same folder with the makefile.
\item Open the makefile with your editor of choice and choose the compiler and flags of choice.
\item Open \hyperlink{SAMSIM_8f90}{SAMSIM.f90}, set a testcase from 1-\/3, and edit the description string to fit your purpose.
\item Use make to compile the code, which produces the executable samsim.x .
\item Make sure a folder \char`\"{}output\char`\"{} is located in the folder with samsim.x .
\item Execute SAMSIM by running samsim.x .
\item Go into output folder
\item Copy the plot script from plotscripts to output
\item Follow the directions written in the plotscripts to plot the output.
\end{DoxyItemize}

Running testcase 4. In contrast to testcase 1-\/3, testcase four requires input files. Input data for testcase is provided in the input folder. Choose one of the subfolders from input/ERA-\/interim/, copy the $\ast$.input files into the folder with the code, and run the executable .samsim.x .

Following modules have a good documentation (both in the code and refman.pdf)
\begin{DoxyItemize}
\item \hyperlink{mo__heat__fluxes_8f90}{mo\_\-heat\_\-fluxes.f90}
\item \hyperlink{mo__layer__dynamics_8f90}{mo\_\-layer\_\-dynamics.f90}
\item \hyperlink{mo__init_8f90}{mo\_\-init.f90}
\end{DoxyItemize}

Biogeochemical tracers can be activated with bgc\_\-flag=2.
\begin{DoxyItemize}
\item Warning! This feature is still relatively new, and has not been standardized yet.
\item The model will track Nbgc number of individual tracer.
\item Especially if you are interested in dissolved gases, you should first make yourself familiar with the bgc\_\-advection subroutine in \hyperlink{mo__mass_8f90}{mo\_\-mass.f90}.
\end{DoxyItemize}

Know issues/Tips and Tricks:
\begin{DoxyItemize}
\item If code changes have no effect, run \char`\"{}make clean\char`\"{} and then \char`\"{}make\char`\"{}, for unknown reasons this is often needed when making changes to \hyperlink{mo__parameters_8f90}{mo\_\-parameters.f90}
\item When bug hunting increase thick\_\-0 and dt, this way the model runs faster, and the output is easier to sort through.
\item Use debug\_\-flag= 2 to output data of each layer at each timestep. Be careful, the output size can become very large!
\item Check dat\_\-settings to keep track of runs, and use the description variable to keep track of experiments.
\item Contact me :)
\end{DoxyItemize}

\begin{DoxyParagraph}{Revision History}
Started by Philipp Griewank 2014-\/05-\/05 
\end{DoxyParagraph}
